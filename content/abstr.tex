\pagenumbering{Roman}
\begin{abstract}{论文~模板}
 摘要是毕业论文(设计)的内容不加注释和评论的简短陈述。摘要主要是说明研究(或设计)工作的目的、方法、结果和结论。摘要应具有独立性和自含性,即不阅读毕业论文(设计),就能获得必要的信息,供读者确定有无必要阅读全文。摘要应用第三人称的方法记述论文的性质和主题,不使用“本文”、“作者”等作为主语,应采用“对…进行了研究”、“报告了…现状”、“进行了…调查”等表达方式。排除在本学科领域已成为常识的内容,不得重复题名中已有的信息。书写要合乎逻辑关系,尽量同正文的文体保持一致。结构要严谨,表达要简明,语义要确切,一般不再分段落。商品名称需要时应加注学名。对某些缩略语、简称、代号等,除了相近专业的读者也能清楚理解的以外,在首次出现处必须加以说明。摘要中通常不用图表、化学结构式以及非公知公用的符号和术语。

毕业论文(设计报告)的摘要包含中文摘要和外文摘要。中文摘要字数为300字以内,外文摘要约为250个实词。

关键词是为了文献标引,从《汉语主题词表》或论文中选取出来,用以表示全文主题内容信息的单词或术语。关键词不宜用非通用的代号和分子式。
关键词的个数为3-8个。关键词的排序,通常应按研究(设计)的对象、性质(问题)和采取的手段排序,关键词后面不加冒号,两词之间应留出一个汉字的空间,不加任何标点符号。
关键词应另起一行,排在摘要的左下方。中外文关键词应一一对应。
\end{abstract}
\begin{abstractEn}{thesis template}
Put your English abtract here.
\end{abstractEn}